\chapter{preliminary}
\section{MPC definition}
- n parties $ p_0,\dots,p_{n-1} $ \\
- agreed functionality F($ x_0,\dots,x_{n-1}. $) \\
- functionality is function with internal randomness 
- party $ p_i $ holds input value $ x_i $ \\
- protocol $ \pi $ that enables the parties to compute  F($ x_0,\dots,x_{n-1}. $) without revelling inputs \\ 





\subsection{the adversary and its capability's.}
 An adversary has the ability to corrupt one or more party's. Their exists an upper bound 0 < t < n. This t limits how many parties the adversary can corrupt and is an important parameter of the setting. A common setting is t = $\left \lfloor{ \frac{n}{2} }\right \rfloor  $, which called the honest majority. For example and for n=3 the presence of an honest majority means, that it is assumed that the adversary can corrupt at most 1 party.    
Once a party is corrupted the adversary get full information about every message the party send our receives, this also includes the messages from the time before the party had been corrupted. There are multiple categorizations of adversary's  and their capability's. On such categorization is the distinction between passive and active adversary's. A passive adversary can not force a corrupted party to deviate from the protocol in an any way. A active adversary has the power to force a corrupted party to deviate from the protocol in an arbitrary way. So if for example the protocol would at some point require that each party choses an integer between 1 and n uniformly at random.  Then a passive adversary would have no choice but to choose the integer between 1 and n uniformly at random. On the contrary an active adversary would be able to force a corrupted party to chose the value 42 or any other value that the adversary considers to be advantageous for him.\\




\subsection{real world vs ideal world }
We can now formalise the question if a protocol $ \pi $ archives our security goal of "not revelling $ x_0,\dots,x_{n-1}. $", this is done by describing an ideal world where there exits a perfect solution for the MPC problem and them comparing the execution of $ \pi $ to this ideal world. In an ideal world there exists an incorruptible third party P that all parties trust. In the real world there is no such incorruptible third party. Instead the parties execute the protocol $ \pi $ be exchanging messages.
In the ideal world evaluating F($ x_0,\dots,x_{n-1}. $) can be done in two steps. In a first round of communication party $ p_i $ send $x_i $ to P.
This gives P all the information required for computing F($ x_0,\dots,x_{n-1} $). In a second round of communication P send each party F($ x_0,\dots,x_{n-1} $).
If F($ x_0,\dots,x_{n-1}) $ is computed this way the ability of the adversary to gain new knowledge about $ x_0,\dots,x_{n-1} $ is minimized. We say that  $ \pi $ is secure against adversary A if A cannot learn more information by attacking $ \pi $ in the real world then by attack the ideal world. This can be shown using simulation based proof. 

The View of a party consist of     
      


   


-simulation based proofs \\
-   