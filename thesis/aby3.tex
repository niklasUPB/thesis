\section{aby3}


\subsection{functionality }
aby3 is a 3 party MPC framework that allows to compute query's on relational database tables. These query's are not limited to queering only one table. Aby3 supports a variety of join operations these include but are not limited to  left join , right join, set union, set minus and also full joins. Note that the semantic of the set union operator in aby3 does not 100 percent matches the semantic of the SQL union operator. 
Similar to the SQL-union the aby3-union does row-concatenation of two tables.  \todo[fancyline]{ todo :union und andere oparatioren die von standart sql abweichen beschreiben}
The result of such join can again be queried using with query's that have a comparable semantic to the SELECT,FROM,WHERE; statement in SQL. Furthermore aby3 comes with a description how aggregate functions like MAX,SUM,COUNT can be realised when utilising aby3.

\subsection{underlying MPC technology}
- honest majority \\ 
- passiv adversary \\
- all protocols constant rounds of communication \\
- O(n) overhead in for join where n is number of rows \\


aby3 demonstrates its capability's in two prototype applications. One of them could be used by the states of the United States to detected voters that are registered in more then one state. What would allow to them to illegitimacy cast a vote in both of these states. This is a showcase example for the use of MPC on very sensitive data that ...





\
In \cite{10.1145/3372297.3423358} ..
-  prototype 
- does feature a LAN VS WAN comparison in benchmarks
- fully  composable
- arbitrary computation computation  