\section {Related Work}
\paragraph{Fundamentals of Secure Multiparty Computation}
As base literature in the field of MPC we have mainly relied on Cramer et al.\cite{cramer2005multiparty} and Lindell \cite{lindell2017simulate}. Cramer et al. gives a broad overview of the topic of secure multiparty computation and its theoretical foundations. Lindell provides an introduction into the topic of simulation based proof which is the key to understanding the security definitions of secure multiparty computation.



\paragraph{ General Purpose Compilers for Secure Multi-Party Computation}
Hastings at al. \cite{hastings2019sok} did also conduct a survey about the state of secure multiparty computation. Yet there are several significant differences between their approach and ours. While Hastings et al. uses general purpose protocols that can compute arbitrary functions.  We evaluate protocols that are specifically designed and optimized for the database context. Their survey was published in 2020, therefore two of the frameworks that are subject of our study, namley ABY3 and VaultDB, have not been published by the time when they conducted their survey. Another difference is that Hastings et al. did not collect data to measure the performance of the frameworks but rather focused on usability and documentation aspects.

\paragraph{MP-SPDZ: A Versatile Framework for Multi-Party Computation}
Keller et al. \cite{hastings2019sok} provide with MP-SPDZ an implementation of over 30 different MPC protocols, which enables it  to compute arbitrary functions. It is notable that MP-SPDZ supports protocols for a range of different security assumptions, these include passive and active adversaries as well as settings in which the adversary may corrupt all but one party alongside with settings where the the adversary can corrupt only a single party. MP-SPDZ comes with a high level programming interface that simplifies its usage. As MP-SPDZ is not optimized for the database operations that are of interest for us it is included in our study. 