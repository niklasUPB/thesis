\section{Implementation}
\begin{lstlisting}[caption={ The Python protocol of Conclave for our first use-case    }]
	def protocol():
	# define the schema for the input tables 
	columns_in_party1 = [
	defCol("primary_key", "INTEGER", [1]),
	defCol("user_data1_Alice", "INTEGER", [1]),
	defCol("user_data2_Alice", "INTEGER", [1]),
	defCol("user_data3_Alice", "INTEGER", [1]),
	]
	# the content of the tables is loaded from a pregenerated .csv
	input_1 = create("input_1", columns_in_party1, {1})
	
	columns_in_party2 = [
	defCol("primary_key", "INTEGER", [2]),
	defCol("user_data1_Bob", "INTEGER", [2]),
	defCol("user_data2_Bob", "INTEGER", [2]),
	defCol("user_data3_Bob", "INTEGER", [2])
	]
	input_2 = create("input_2", columns_in_party2, {2})
	# calculate the join over the two tables 
	join_result = join(input_1, input_2, 'join_result', ['primary_key'], ['primary_key'])
	# reveal the output of the join to Alice
	collect(join_result, 1)
	# reveal the output of the join  to Bob
	collect(join_result, 2)
	
	if __name__ == "__main__":
	with open(sys.argv[1], "r") as config:
	# load the configuration data
	config = json.load(config)
	# tell Conclave to generate secure code for the protocol an execute it
	workflow.run(protocol, c, mpc_framework="jiff", apply_optimisations=True)
\end{lstlisting}
\begin{lstlisting}[caption={Simpifiyed Protocol for our first use-case in ABY3}]
	
	std::vector<ColumnInfo> AliceCols = { ColumnInfo{ "key", TypeID::IntID, keyBitCount } };
	std::vector<ColumnInfo> BobCols = { ColumnInfo{ "key", TypeID::IntID, keyBitCount } };
	
	for (u32 i = 1; i < cols; ++i)
	{
		AliceCols.emplace_back("Alice" + std::to_string(i), TypeID::IntID, 32);
		BobCols.emplace_back("Bob" + std::to_string(i), TypeID::IntID, 32);
	}
	# Create tables for Alice and Bob and fill them with content
	Table AliceTable(rows, AliceCols);
	Table BobTable(rows, BobCols);
	# Fill the primary columns 
	for (u64 i = 0; i < rows; ++i)
	{
		# if out is false then the entry will be included in the join
		auto out = (i >= intersectionsize);
		for (u64 j = 0; j < 2; ++j)
		{
			AliceTable.mColumns[0].mData(i, j) = i + 1;
			BobTable.mColumns[0].mData(i, j) = i + 1 + (rows * out);
		}
	}
	# Fill the other columns with random integers
	for (u64 i = 1;  i < cols; ++i){
		for (u64 j =0; j < rows; ++j){
			AliceTable.mColumns[i].mData(j, 0) = rand() ;
			BobTable.mColumns[i].mData(j, 0) = rand();	
		}
	}
	# instanciate Timer for benchmarking
	Timer t;
	# Alice and Bob each run their own thread 
	auto routine = [&](int i) { setThreadName("t0");
		t.setTimePoint("start");
		# 
		auto A = (i == 0) ? srvs[i].localInput(AliceTable) : srvs[i].remoteInput(0);
		auto B = (i == 0) ? srvs[i].localInput(BobTable) : srvs[i].remoteInput(0);
		
		if (i == 0) t.setTimePoint("inputs");
		if (i == 0) srvs[i].setTimer(t);
		std::vector<SharedTable::ColRef>  First_Select_collumns;		
		for (u64 i = 0;  i < cols; ++i){
			First_Select_collumns.emplace_back(SharedTable::ColRef(B,B.mColumns[i]) );
		}
		for (u64 i = 1;  i < cols; ++i){
			First_Select_collumns.emplace_back(SharedTable::ColRef(A,A.mColumns[i]) );
		}
		auto result =srvs[i].join( SharedTable::ColRef(A,A.mColumns[0]) , SharedTable::ColRef(B,B.mColumns[0]), First_Select_collumns);
		if (i == 0) t.setTimePoint("intersect");
		for (u64 index = 0; index < result.mColumns.size(); ++index)
		{
			aby3::i64Matrix reveal(result.mColumns[index].rows(),  result.mColumns[index].i64Cols());
			server[i].mEnc.revealAll(server[i].mRt.mComm, T.mColumns[index], reveal);
			if (i == 0) std:: cout << reveal << std::endl;	
		}
	};
	
	auto t0 = std::thread(routine, 0);
	auto t1 = std::thread(routine, 1);
	t0.join();
	t1.join();
}
\end{lstlisting}