\section{conclave}


\subsection{functionality}
Conclave is a framework that allows two or three parties to perform MPC analytics on "big data". Conclave aims provide a high-level interface that abstracts internal MPC details away from the user. Every operation done with conclave is composable, that means that the output of every query can be the input of another query. This mechanism makes it possible to construct very complex queries out of multiple relative simple queries.  With conclave one can join tables using the equivalent of an equi-join or an union operator. Conclave also supports a range of aggregate functions these include sum, mean, standard deviation. 

\subsection{underlying MPC technology}
- relying on Obliv-C and Sharemind as MPC backend\\
- inherits security guarantees of backend -> honest majority \\
- secure against semi-honest adversary \\
- adversary is statically\\
- no secure channel setting\\


\subsection{querie optimization}
- combining MPC operations and cleartext processing\\ 
- moving operations outside of MPC to maximise performance\\
- maintaining same end-to-end security as "pure" MPC     \\
- contrary to conventional sql operations that aims to minimize the total amount of work e.g. filters before join \\


- published in 2019,\\
- allows to explicitly annotate trust between parties\\
- utilises trust through hybrid operations for additional performance increase \\
- backend theoretic exchangeable through generic interface\\
- compares to "SMCQL most similar existing system"\\
- jiff dependency \\
- requires python 3.5 \\
- ...