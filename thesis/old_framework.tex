\section{Rejected Frameworks}
\paragraph{CipherCompute}
On candidate for our study was CipherCompute \cite{Cosmian} . With the CypherCompute framework it is possible to solve a huge range of MPC problems using Rust. These include SQL operations like joins that are of interest for us. Furthermore CypherCompute provides a rich documentation, consisting of a full quickstart guide and several well documented example projects. CypherCompute utilises the SCALE-MAMBA \cite{aly2021scale} framework for its underlying MPC operations. SCALE-MAMBA  itself has evolved out of the well-known SPDZ \cite{SPDZ} protocol. Unfortunately the early access version of CypherCompute is out of maintenance by the time we conducting this study. Therefore we have decided to not include CypherCompute in our study.

\paragraph{Prio+}
Prio+ \cite{cryptoeprint:2021:576} is the next generation of the highly influential Prio \cite{201553}. Prio+ strives to maintain the same use and security as Prio, while significantly increasing performance compared to its predecessor. Prio Plus allows an arbitrary number of parties to jointly compute aggregated statistics, like SUM, MAX, MIN operators. Prio+ utilises a client server model. In which the (potentially many) input parties use a small number of servers to compute the statistics. Prio+ guarantees confidentiality of the input values if at least one server stays honest. Unlike CipherCompute or Conclave Prio+ is not a framework for developing MPC solutions. Its rather a system for special purposes. This means that the use of Prio+ can not be extended beyond the usecases that have been originally implemented by the authors of Prio+. This leaves Prio+ with a relatively small range of usecases compared to frameworks like ABY3 or Conclave. Therefore we have decided to not include Prio+ in our study.

\paragraph{VaultDB}
VauldDB \cite{rogers2022vaultdb} 
- usese EMP toolkit \cite{emp-toolkit}
- demonstrates proof of concept  