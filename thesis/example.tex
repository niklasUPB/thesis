\chapter{Basic definitions and notation}
\todo[inline]{This file contains example content. It is meant to get you started.}
\todo[inline]{You can remove this example content from the thesis by removing the input statement for example.tex from the thesis\_main.tex file.}

\section{Basic notation}
\label{sec:basicnotation}
Throughout this thesis, we will use the following notation:
\begin{itemize}
	\item $\N := \{1,2,\dots\}$ denotes the set of natural numbers (excluding zero). %see config/defs.tex file for the definition of the $\N$ macro.
	\item For a bit string $s = (s_0,\dots,s_{n-1}) \in \{0,1\}^n$ and $0\leq i \leq j < n$, we write $s[i:j] := (s_i, \dots, s_{j-1})$ to denote substrings.
	In particular, $s[i:i]$ is the empty string $\epsilon$ and $s[0:n]$ is the complete string $s$. 
	\item For two vectors $\vec{u},\vec{v}\in \{0,1\}^n$, with $\vec{u}=(u_1,\dots,u_n), \vec{v} = (v_1,\dots,v_n)$, the expression $\vec{u}\odot\vec{v}$ denotes the Hadamard product. $(\vec{u}\odot\vec{v})_i = u_i\cdot v_i$.
\end{itemize}

\section{Gobbling schemes}
\todo[inline]{Don't try to make sense of this. It's just a syntax example with no real connection to anything.}
Gobbling schemes \todo{This is an example for a todonote. They're super useful!} are useful whenever the the Hadamard product of two random bit vectors needs to be hidden from polynomially bounded adversaries. 
They are an important building block for twaddle signatures, which we investigate in this thesis.
\subsection{Syntax definition}
Our definition for gobbling schemes is taken from \cite{testref} \todo[fancyline]{note the citation} with minor syntactic changes. 
See Section~\ref{sec:basicnotation} \todo[fancyline]{note the reference} for basic notation.

\begin{definition}[Gobbling scheme]
	A \emph{gobbling scheme} $\Pi$ consists of the following three probabilistic polynomial-time algorithms:
	\begin{itemize}
		\item $\pk \leftarrow \setup(1^\lambda)$ \todo{note that $\pk$ and $\setup$ are macros defined in defs.tex.} 
			on input a unary security parameter $\lambda$, $\setup$ generates a public key $\pk$.
		\item $\vec{q} \leftarrow \gobble(\pk, \vec{u}, \vec{v})$ generates a \emph{gobbled} vector $\vec{q}\in\{0,1\}^n$ given a key $\pk$ and two vectors $\vec{u},\vec{v}\in\{0,1\}^n$.
		\item $\vec{z} \leftarrow \ungobble(\pk, \vec{q})$ given a gobbled vector $\vec{q}\in\{0,1\}^n$ outputs an ungobbled vector $\vec{z}\in\{0,1\}^{2n}$.
	\end{itemize}
	$\Pi$ is \emph{correct} if there exists a negligible function $\mu$ such that for all $\lambda,n\in\N$,
	\begin{align*}
		1-\mu(\lambda) \leq \Pr[\vec{z} = \vec{u}\odot\vec{v} \mid & k\leftarrow \setup(1^\lambda); \vec{u},\vec{v}\leftarrow \{0,1\}^n; \\
		& \vec{z} \leftarrow \ungobble(k, \gobble(k, \vec{u},\vec{v}))]
	\end{align*}
\end{definition}
\noindent Intuitively, correctness guarantees \dots

\subsection{Security definition}
\dots
