\chapter {Introduction}
Secure multiparty computation is subfield of modern cryptography that focuses around the question of how multiple parties with private inputs can evaluate a function over these inputs in a secure manner. In the context of multiparty computation secure means that no information about the private inputs is revealed. The first problem that was studied in the context of secure multiparty computation was Yao's ``Millionaire’s Problem''. In Yao's ``Millionaire’s Problem'' there are two millionaires that want to know who of them is wealthier and do not want to disclosure any other information about their wealth. The ``Millionaire’s Problem'' was solved by Andrew Yao in 1982 \cite{4568388} using Garbled Circuits.
\paragraph{Secure Multiparty Computation}
In the generally secure multiparty computation setting(MPC), there are n parties $ P_1,\dots,P_{n} $ that want to compute an agreed upon public function F based on their inputs $ x_1,\dots,x_{n}$. It is desired that no party $P_i$ learns anything about the inputs of the other parties, that is not revealed by his input and the result of the function. In the example of the ``Millionaire’s Problem'' there are two parties $P_0,P_1 $, therefore n equals 2 and the public function is $F=argmax(x_1,x_2)$. In the context of the ``Millionaire’s Problem'' secure means that $P_1$ and $P_1$ only learn who of $P_1$ and $P_2$ is wealthier, i.e they only learn if $x_1$ > $x_2$ holds or $x_1$ < $x_2$ holds, as that cannot be avoided if they want to evaluate the function. 
There is a variety of different security models for the field of secure multiparty computation, that characterise the attacker or adversary and its capabilities. For example an attacker may be able to corrupt at most one party or possible many parties. Another important security model is the distinction between a passive adversary and an active adversary, an active adversary is allowed to force a corrupted party to change its behaviour, while a passive adversary can only observe it. 
For the majority of security models there already exist generic solutions that allow to evaluate an arbitrary probabilistic polynomial time function under these models. Such generic protocols like \cite{DBLP:conf/crypto/WigdersonD82} and \cite{4568388} have shown to have insufficient efficiency to be applied in practice. Therefore efforts have been made to develop new, more sophisticated protocols and optimise existing solutions.
Archer et al. \cite{Archer2018FromKT} and Hastings et al. \cite{hastings2019sok} have conducted an evaluation of existing multiparty computation frameworks. Yet, their studies are not exhaustive and are partly concerned with simple toy problems.
Therefore, they provide only limited information about the usability of secure multiparty computation in real world applications.  
\paragraph{Database Applications}
One use case for secure multiparty computation are large scale database systems, where multiple companies may own large databases and are willing to cooperate together and compute shared statistics over their combined data. Yet they cannot afford to disclose individual records, as these may contain highly sensitive personal information or highly valuable trade secrets. One such use case is described by Bater et al. \cite{bater}, where several hospitals work together to use their patient data for scientific research. Of course, due to legal regulations, the individual patient data must remain strictly confidential. Another example for a practical utilisation of secure multiparty computation is documented by Archer et al. \cite{Archer2018FromKT}, where in a pilot project, the secure multiparty computation framework Sharemind \cite{bogdanov2015estonian} was utilized to implement a system that helped the Estonian government detect tax fraud. Since the individual data sets cannot be published, join and aggregate operations are of particular importance. Frameworks that are included in our study are ABY3 \cite{mohassel2020fast}, CipherCompute\footnote{https://github.com/Cosmian/CipherCompute} , Conclave \cite{10.1145/3302424.3303982}, Prio+ \cite{cryptoeprint:2021:576} , SMCQL \cite{bater} and VaultDB \cite{rogers2022vaultdb}. These frameworks are a very heterogeneous group and differ in their capabilities and security assumptions. For example Prio+ does support an arbitrary number of parties, while VaultDB only supports two and ABY3 works exclusively in a three party setting.












%\section{MPC and Databases}
%In 2018 Archer et al.[ABL+18] published a case study that studied to what degree secure multiparty computation (MPC) is applicable to real-world problems. Since 2018 the field of MPC saw a lot of change. In our thesis, we want to take a closer look at some of the newer MPC frameworks and evaluate their performance and usability. 
%\paragraph{Secure Multi Party Computation}
%A famous problem in the context of MPC is Yao's millionaire's problem. In Yao's millionaire's problem there are two millionaires Alice and Bob. Alice and Bob want to know who of them is has more money. i.e. they want to compute the function F($AliceMoney,BobMoney$) := $ 
%\begin{dcases} 
%	1  &  AliceMoney \leq  BobMoney \\
%	0  &  BobMoney > AliceMoney  \\
%\end{dcases}  $. Yet neither of them is willing to trust the other and tell him how much money he has. 
%Yao's millionaire's problem can be generalised into the general MPC problem.  Instead of Bob and Alice, we now consider n parties $ p_0,\dots,p_{n-1} $ and each party i holds an arbitrary input  $ x_i $ for an arbitrary function F($ x_0,\dots,x_{n-1}$),  that all parties have agreed upon.  
%A secure multiparty computation protocol $ \pi $  is a protocol, that allows  $ p_0,\dots,p_{n-1} $ to compute F($x_0,\dots,x_{n-1} )$ without revelling any information about $ x_0,\dots,x_{n-1}. $
%\paragraph{Database Applications}
%A more modern and complex use case for MPC are aggregated statistics over large
%datasets and shared databases. In this scenario, two or more input parties each hold
%a share of a large database and they want to combine these secret shares to a jointed
%database that can be queried without leaking information over their shares. For some, secure multiparty computation may sound like a technique of only theoretical interest. Yet there are many instances %where such systems based on MPC have been
%able to solve real-world problems. One such instance is documented in [ABL+18]. In a pilot project, the MPC framework Sharemind was utilized to implement a system that helped the Estonian government to detect tax fraud.





             
%Andrew Yao proposed a solution for Yao's millionaire's problem in 1982 [].  It has also been shown that MPC is Turing-complete[]. This means that for any function f that can be computed with a Turing machine. There exists a MPC protocol $ \pi $ that can compute f.    
%-Databases ...








