\chapter {Introduction}
\section{MPC and Databases}
In 2018 Archer et al.[ABL+18] published a case study that studied to what degree secure multiparty computation (MPC) is applicable to real-world problems. Since 2018 the field of MPC saw a lot of change. In our thesis, we want to take a closer look at some of the newer MPC frameworks and evaluate their performance and usability. 
\paragraph{Secure Multi Party Computation}
A famous problem in the context of MPC is Yao's millionaire's problem. In Yao's millionaire's problem there are two millionaires Alice and Bob. Alice and Bob want to know who of them is has more money. i.e. they want to compute the function F($AliceMoney,BobMoney$) := $ 
\begin{dcases} 
	1  &  AliceMoney \leq  BobMoney \\
	0  &  BobMoney > AliceMoney  \\
\end{dcases}  $. Yet neither of them is willing to trust the other and tell him how much money he has. 
Yao's millionaire's problem can be generalised into the general MPC problem.  Instead of Bob and Alice, we now consider n parties $ p_0,\dots,p_{n-1} $ and each party i holds an arbitrary input  $ x_i $ for an arbitrary function F($ x_0,\dots,x_{n-1}$),  that all parties have agreed upon.  
A secure multiparty computation protocol $ \pi $  is a protocol, that allows  $ p_0,\dots,p_{n-1} $ to compute F($x_0,\dots,x_{n-1} )$ without revelling any information about $ x_0,\dots,x_{n-1}. $
\paragraph{Database Applications}
A more modern and complex use-case for MPC are aggregated statistics over large
datasets and shared databases. In this scenario, two or more input parties each hold
a share of a large database and they want to combine these secret shares to a jointed
database that can be queried without leaking information over their shares. For some, secure multiparty computation may sound like a technique of only theoretical interest. Yet there are many instances where such systems based on MPC have been
able to solve real-world problems. One such instance is documented in [ABL+18]. In a pilot project, the MPC framework Sharemind was utilized to implement a system that helped the Estonian government to detect tax fraud.





             
%Andrew Yao proposed a solution for Yao's millionaire's problem in 1982 [].  It has also been shown that MPC is Turing-complete[]. This means that for any function f that can be computed with a Turing machine. There exists a MPC protocol $ \pi $ that can compute f.    
%-Databases ...








