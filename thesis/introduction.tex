\chapter {Introduction}
\section{MPC and Databases}
A famous problem in the context of MPC is Yao's millionaire's problem. In Yao's millionaire's problem there are two millionaires Alice and Bob. We will call Alice's wealth x and Bob's wealth y. Alice and Bob want to know who of them is has more money. i.e. they want to compute the function F(x,y) := $ 
\begin{dcases} 
	Alice \,  is \, richer   &  y \leq  x \\
	Bob \, is \, richer   &  y > x  \\
\end{dcases}  $. Yet neither of them is willing to trust the other and tell him how much money he has. 
Yao's millionaire's problem can be generalised into the general MPC problem.  Instead of Bob and Alice, we now consider n parties $ p_0,\dots,p_{n-1} $ and each party i holds an arbitrary input  $ x_i $ for an arbitrary function F($ x_0,\dots,x_{n-1}$),  that all parties have agreed upon.  
A MPC protocol $ \pi $  is protocol, that allows  $ p_0,\dots,p_{n-1} $ to compute F($ x_0,\dots,x_{n-1} )$ without revelling any information about $ x_0,\dots,x_{n-1}. $
       

             
Andrew Yao proposed a solution for Yao's millionaire's problem in 1982 [].  It has also been shown that MPC is Turing-complete[]. This means that for any function f that can be computed with a Turing machine. There exists a MPC protocol $ \pi $ that can compute f.    
-Databases ...








