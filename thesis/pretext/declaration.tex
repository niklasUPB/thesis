\newgeometry{top=10mm, bottom=20mm, right=20mm}
\begin{small}

  \parindent0ex

  \chapter*{Eidesstattliche Versicherung}

  \thispagestyle{empty}

  \bigskip

  Nachname: \parbox[t]{.343\linewidth}{\rule[-3pt]{\linewidth}{.4pt}} \hspace{2ex} Vorname: \parbox[t]{.35\linewidth}{\rule[-3pt]{\linewidth}{.4pt}}

  \vspace{2\bigskipamount}

  Matrikelnr.: \parbox[t]{.15\linewidth}{\rule[-3pt]{\linewidth}{.4pt}} \hspace{2ex} Studiengang: \parbox[t]{.5\linewidth}{\rule[-3pt]{\linewidth}{.4pt}}

  \bigskip

  $\square$ Bachelorarbeit \hspace{1ex} $\square$ Masterarbeit

  \bigskip

  Titel der Arbeit: \Title

  \bigskip

  $\square$ Die elektronische Fassung ist der Abschlussarbeit beigefügt.

  \bigskip

  $\square$ Die elektronische Fassung sende ich an die/den erste/n Prüfenden bzw.~habe ich an die/den erste/n Prüfenden gesendet.

  \bigskip

  Ich versichere hiermit an Eides statt, dass ich die vorliegende Abschlussarbeit (Ausarbeitung inkl.~Tabellen, Zeichnungen, etc.)~selbstständig und ohne unzulässige fremde Hilfe erbracht habe.
  Ich habe keine anderen als die angegebenen Quellen und Hilfsmittel benutzt sowie wörtliche und sinngemäße Zitate kenntlich gemacht.
  Die Abschlussarbeit hat in gleicher oder ähnlicher Form noch keiner Prüfungsbehörde vorgelegen.
  Die elektronische Fassung entspricht der gedruckten und gebundenen Fassung.

  \subsection*{Belehrung}

  Wer vorsätzlich gegen eine die Täuschung über Prüfungsleistungen betreffende Regelung einer Hochschulprü\-fungsordnung verstößt, handelt ordnungswidrig.
  Die Ordnungswidrigkeit kann mit einer Geldbuße von bis zu 50.000,00 \euro~geahndet werden.
  Zuständige Verwaltungsbehörde für die Verfolgung und Ahndung von Ordnungswidrigkeiten ist die Vizepräsidentin / der Vizepräsident für Wirtschafts- und Personalverwaltung der Universität Paderborn.
  Im Falle eines mehrfachen oder sonstigen schwerwiegenden Täuschungsversuches kann der Prüfling zudem exmatrikuliert werden.~(§~63 Abs.~5 Hochschulgesetz NRW in der aktuellen Fassung).

  \medskip

  Die Universität Paderborn wird ggf.~eine elektronische Überprüfung der Abschlussarbeit durchführen, um eine Täuschung festzustellen.

  \medskip

  Ich habe die oben genannten Belehrungen gelesen und verstanden und bestätige dieses mit meiner Unterschrift.

  \vspace{2\bigskipamount}

  Ort: \parbox[t]{.4\linewidth}{\rule[-3pt]{\linewidth}{.4pt}} \hspace{2ex} Datum: \parbox[t]{.15\linewidth}{\rule[-3pt]{\linewidth}{.4pt}}

  \vspace{2\bigskipamount}

  Unterschrift: \parbox[t]{.5\linewidth}{\rule[-3pt]{\linewidth}{.4pt}}

  \subsection*{Datenschutzhinweis}

  Die o.g.~Daten werden aufgrund der geltenden Prüfungsordnung (Paragraph zur Abschlussarbeit) i.V.m.~§~63 Abs.~5 Hochschulgesetz NRW erhoben.
  Auf Grundlage der übermittelten Daten (Name, Vorname, Matrikelnummer, Studiengang, Art und Thema der Abschlussarbeit) wird bei Plagiaten bzw.~Täuschung der/die Prüfende und der Prüfungsausschuss Ihres Studienganges über Konsequenzen gemäß Prüfungsordnung i.V.m.~Hochschulgesetz NRW entscheiden.
  Die Daten werden nach Abschluss des Prüfungsverfahrens gelöscht.
  Eine Weiterleitung der Daten kann an die/den Prüfende/n und den Prüfungsausschuss erfolgen.
  Falls der Prüfungsausschuss entscheidet, eine Geldbuße zu verhängen, werden die Daten an die Vizepräsidentin für Wirtschafts- und Personalverwaltung weitergeleitet.
  Verantwortlich für die Verarbeitung im regulären Verfahren ist der Prüfungsausschuss Ihres Studiengangs der Universität Paderborn, für die Verfolgung und Ahndung der Geldbuße ist die Vizepräsidentin für Wirtschafts- und Personalverwaltung.

\end{small}

\restoregeometry
